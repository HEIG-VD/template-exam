\titledquestion{Choix multiples}[10]

\begin{parts}

    \part Exprimez l'équation de Friedmann pour un univers homogène et isotrope en termes de la densité d'énergie totale \(\rho\), la constante de Hubble \(H\), et la constante de courbure \(k\).
    \begin{solutionordottedlines}[3cm]
        $$H^2 = \frac{8\pi G}{3}\rho - \frac{k}{a^2}$$
    \end{solutionordottedlines}

    \part Quel est le paramètre de densité critique \(\Omega\) pour un univers plat et comment se définit-il en termes de la densité d'énergie \(\rho\) et de la densité critique \(\rho_c\) ?
    \begin{solutionordottedlines}[3cm]
        $$\Omega = \frac{\rho}{\rho_c} = 1$$
    \end{solutionordottedlines}

    \part Décrivez brièvement le processus de l'inflation cosmique et mentionnez une des principales raisons pour lesquelles cette théorie a été introduite pour expliquer les observations cosmologiques.
    \begin{solutionordottedlines}[3cm]
        L'inflation cosmique est une période de croissance exponentielle extrêmement rapide de l'univers, proposée pour résoudre les problèmes de l'horizon et de la platitude.
    \end{solutionordottedlines}

    \part Quelle est l'équation de l'état pour une composante de l'univers dominée par l'énergie noire et comment se caractérise-t-elle en termes du paramètre de pression \(w\) ?

    \begin{solutionordottedlines}[3cm]
        $$p = w\rho \, \text{où} \, w \approx -1$$
    \end{solutionordottedlines}

    \part Dans le cadre du modèle de \(\Lambda\)CDM, quelle est la relation entre l'âge de l'univers \(t_0\) et la constante de Hubble \(H_0\) en l'absence de toute autre composante énergétique ?

    \begin{solutionordottedlines}[3cm]
        $$t_0 \approx \frac{1}{H_0}$$
    \end{solutionordottedlines}

\end{parts}