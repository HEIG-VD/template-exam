
\titledquestion{Quiz}[10]



\question Considérez un champ scalaire quantique \(\phi(x)\) dans une théorie \(\phi^4\) avec l'action suivante :
\[ S[\phi] = \int d^4x \left( \frac{1}{2} \partial_\mu \phi \partial^\mu \phi - \frac{1}{2} m^2 \phi^2 - \frac{\lambda}{4!} \phi^4 \right) \]
Quel est le propagateur de Feynman pour ce champ scalaire dans l'espace des impulsions ?
\begin{parts}
\part
\begin{multicols}{2}
\begin{checkboxes}
    \choice \(\frac{1}{p^2 - m^2 + i\epsilon}\)
    \choice \(\frac{1}{p^2 + m^2 + i\epsilon}\)
    \choice \(\frac{1}{p^2 - m^2 - i\epsilon}\)
    \choice \(\frac{1}{p^2 + m^2 - i\epsilon}\)
\end{checkboxes}
\end{multicols}
\end{parts}

\question Considérez la métrique de Schwarzschild décrivant l'extérieur d'un trou noir de masse \(M\) :
\[ ds^2 = -\left(1 - \frac{2GM}{r}\right)dt^2 + \left(1 - \frac{2GM}{r}\right)^{-1}dr^2 + r^2(d\theta^2 + \sin^2\theta \, d\phi^2) \]
Quelle est la condition pour que le rayon \(r\) soit un rayon de Schwarzschild (horizon des événements) ?
\begin{parts}
\part
\begin{multicols}{2}
\begin{checkboxes}
    \choice \(r = GM\)
    \choice \(r = \frac{GM}{c^2}\)
    \choice \(r = 2GM\)
    \choice \(r = \frac{2GM}{c^2}\)
\end{checkboxes}
\end{multicols}
\end{parts}

\question Dans l'approximation de Born-Oppenheimer pour une molécule, l'équation de Schrödinger électronique est donnée par :
\[ \hat{H}_{elec} \psi_{elec}(\mathbf{r}; \mathbf{R}) = E_{elec}(\mathbf{R}) \psi_{elec}(\mathbf{r}; \mathbf{R}) \]
Laquelle des affirmations suivantes décrit correctement cette équation ?
\begin{parts}
\part
\begin{multicols}{2}
\begin{checkboxes}
    \choice \(\hat{H}_{elec}\) inclut les mouvements des noyaux.
    \choice \(\psi_{elec}(\mathbf{r}; \mathbf{R})\) dépend uniquement des coordonnées électroniques \(\mathbf{r}\).
    \choice \(E_{elec}(\mathbf{R})\) est l'énergie totale de la molécule.
    \choice \(\hat{H}_{elec}\) inclut l'interaction électron-noyau et électron-électron.
\end{checkboxes}
\end{multicols}
\end{parts}